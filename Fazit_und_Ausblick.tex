% Die Arbeit besteht aus Kapiteln (chapter)
\chapter{Zusammenfassung und Schlusswort}

\section{Zusammenfassung}
Nochmal im Überblick: Was sind Möglichkeiten und Grenzen von FleX und Game Engine? Was musste hinzugefügt werden?

\section{Fazit und Ausblick}
Die für diese Arbeit gesetzten Ziele konnten größtenteils erreicht werden?

...Zwar sind solche Simulatoren unbestreitbar weniger realistisch, dafür sinken die nötigen Kosten weil keine haptische Hardware benötigt wird. Zumal heutige VR-Geräte für immer mehr Privatpersonen erschwinglich sind. Das in dieser Arbeit verwendete VR-Gerät, die Oculus Quest 2, kostete beispielsweise 300 Euro...

... weil FleX inzwischen auch mit DX funktioniert, könnten Simulatoren auch als stand alone variante portiert werden. Dann wird NUR  noch Oculus Quest 2 und kein pc mehr benötigt... allerdings bräuchte man dann ue4.22 mindestens...


aus kapitel attachement:
Erfreulicherweise unterstützt die verwendete Unreal Integration bereits das Befestigen von Szenen oder anderen 3D Objekten an ein Flex-Objekt. Allerdings drehen sich Szenen nicht mit Soft Bodies mit. Nur die Position der befestigten Szenen wird angepasst, während die Rotation unverändert bleibt. 
Diese Problemstellung wurde im Rahmen dieser Arbeit nicht weiter verfolgt, weil hier keine Szenen an Soft Bodies befestigt werden mussten. Es wurden nur Szenen an Rigid Bodies befestigt, was problemlos funktioniert.

Diese Problemstellung könnte interessant für nachfolgende Arbeiten sein.

% ----------------------------------- ALT --------------------------------------------

%%\section{Fazit}
%
%Im Rahmen dieser Arbeit konnten fast alle gesetzten Ziele erreicht werden. Anhand des konzipierten Simulationsmodells konnte ein Prototyp implementiert werden, der fast alle Anforderungen aus Kapitel \ref{sec_Anforderungen} erfüllt. 
%
%Der im Prototyp simulierte Schwarm kann durch Einsatz einfacher Zusatzverhaltensregeln von einem menschlichen Spieler gejagt oder verfolgt werden. Das bestehende System lässt sich zukünftig mit neuen Verhaltensregeln erweitern. Beispielsweise könnte ein Spieler ein Flussfeld zeichnen, um virtuelle \glqq Autobahnen\grqq{} für die Boids zu generieren. Gerät ein Individuum auf solch eine Bahn, folgt es dem Flussfeld, so wie Autos einer Autobahn folgen.
%
%Die Bewegungen des Schwarms können bei entsprechender Parametrisierung realistisch und lebendig gestaltet werden. Ohne mit zufällig generierten Steuervektoren arbeiten zu müssen, bewegen sich manche Schwärme scheinbar in zufällige Richtungen, als ob die virtuellen Lebewesen ein eigenes Ziel verfolgen würden.
%
%Die Performance des implementierten Prototyps erwies sich, entgegen aller Erwartungen, als überraschend gut. Die Messungen aus Kapitel \ref{sec_performance} lassen vermuten, dass sich das konzipierte Simulationsmodell durchaus in der Praxis anwenden lässt. Künftige Videospiele könnten von diesen interaktiven Schwarm-Wesen profitieren, zumal sich die entwickelte Simulation von herkömmlichen Shadern ausgeführt werden kann. Shader-Programme, die mithilfe der Unreal Engine 4 implementiert wurden, lassen sich ohne große Umstände auf einige Plattformen portieren. Von allen handelsüblichen Konsolen, bis hin zu Smartphones. Smartphone-Besitzer mithilfe des Touchscreens mit dem Schwarm interagieren zu lassen, ist durchaus umsetzbar. 
%
%Der erstellte Prototyp konnte aus zeitlichen Gründen nur als zweidimensionale Schwarmsimulation implementiert werden. Die Idee, den virtuellen Schwarm als dreidimensionale Leuchtpunkt-Wolke, mittels volumetrischer Render-Methodik, darzustellen, wird aber weiterhin mit großem Interesse verfolgt. Zu diesem Zeitpunkt ist jedoch noch nicht bekannt, wann die Unreal Engine 4 dreidimensionale Textur-Formate unterstützen wird.
%
%
%
%%\section{Ausblick}
%
%Trotz allem ist das bestehende System durchaus verbesserungswürdig. Besonders negativ aufgefallen ist der in Kapitel \ref{sec_LoseBoids} angesprochene Verlust von Boids. Diese Problematik konnte aus Zeitgründen nicht weiter untersucht werden, weswegen noch keine konkrete Erklärung für das Verschwinden von Individuen existiert.
%
%Außerdem könnte die Darstellung des Schwarms weiter verbessert werden. Die Schwarm-Textur wirkt zum Teil noch recht körnig und unregelmäßig. Es entstehen auch hin und wieder harte Kannten und Ecken.
%
%Eine weitere interessante Erweiterung wäre, die Wahrnehmungsvektoren der Boids wie folgt einzusetzen:  Wie in der Robotik üblich, könnten die Wahrnehmungssensoren nur an die Vorderseite eines Boids befestigt werden. Dann würde jedes Individuum nur den Bereich wahrnehmen können, der in dessen Bewegungsrichtung liegt. Das könnte den Realismus der Simulation steigern, weil dann die Bewegungsrichtung eines Boids auch seine \glqq Blickrichtung\grqq{} wäre.
%
%Denkt man über die Funktionsweise des konzipierten Modells genauer nach, lässt sich eine interessante Feststellung machen. In diesem Simulationsmodell entsteht keine Separation oder Kohäsion weil benachbarte Boids einen zu geringen oder zu großen Abstand zueinander haben. In diesem System handeln die Individuen dann wenn in einer benachbarten Zelle eine bestimmte Menge Boids enthalten ist. Hier ist also nicht die Entfernung, sondern vielmehr die Menge ausschlaggebend. Nichts desto trotz können auch in diesem System besonders realistisch wirkende Sachwarmbewegungen entstehen.
%
%Zum Schluss lässt sich sagen, dass sich das geniale Prinzip von Schwärmen erwiesen hat: Obwohl jede Zelle des Zellensystems nur wenige einfache Fähigkeiten beherrscht, entstand am Ende ein Gesamtsystem, das sich viel komplexer als seine einzelnen Bausteine verhalten kann. Jedes Boid kann nur wenige Bewegungsrichtungen einschlagen und besitzt sehr eingeschränkte Wahrnehmungsfähigkeiten. Aber dennoch kann sich eine große Gruppe aus Boids scheinbar in alle möglichen Richtungen bewegen und dabei einzigartige Formationen bilden.
%
