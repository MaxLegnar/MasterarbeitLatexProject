% -------------------------------------------------------
% Daten für die Arbeit
% Wenn hier alles korrekt eingetragen wurde, wird das Titelblatt
% automatisch generiert. D.h. die Datei titelblatt.tex muss nicht mehr
% angepasst werden.

\newcommand{\hsmasprache}{de} % de oder en für Deutsch oder Englisch

% Titel der Arbeit auf Deutsch
\newcommand{\hsmatitelde}{Medizinische Simulatoren in virtueller Realität ohne haptisches Feedback, mit Position Based Dynamics und Game Engines}

% Titel der Arbeit auf Englisch
\newcommand{\hsmatitelen}{Medical simulators in Virtual Reality without haptic feedback, using Position Based Dynamics and Game Engines}
%implementation of a buoyancy simulation and an interactive fluid surface for video games

% Weitere Informationen zur Arbeit
\newcommand{\hsmaort}{Heidelberg}    % Ort
\newcommand{\hsmaautorvname}{Maximilian} % Vorname(n)
\newcommand{\hsmaautornname}{Legnar} % Nachname(n)
\newcommand{\hsmaautornmatr}{3544557} % Nachname(n)
\newcommand{\hsmadatum}{30.04.2021} % <----------Datum der Abgabe 01.11.2020 - 30.04.2021
\newcommand{\hsmajahr}{2021} % Jahr der Abgabe
\newcommand{\hsmafirma}{Ruprecht-Karls-Universität Heidelberg} % Firma bei der die Arbeit durchgeführt wurde
\newcommand{\hsmabetreuer}{Prof. Dr. Jürgen Hesser} % Betreuer an der Hochschule
\newcommand{\hsmazweitkorrektor}{Prof. Dr. Susanne Krömker} % Betreuer im Unternehmen oder Zweitkorrektor
\newcommand{\hsmafakultaet}{I} % I für Informatik
\newcommand{\hsmastudiengang}{MEB} % IB IMB UIB IM MTB

% Zustimmung zur Veröffentlichung
\setboolean{hsmapublizieren}{true}   % Einer Veröffentlichung wird zugestimmt
\setboolean{hsmasperrvermerk}{false} % Die Arbeit hat keinen Sperrvermerk

% -------------------------------------------------------
% Abstract

% Kurze (maximal halbseitige) Beschreibung, worum es in der Arbeit geht auf Deutsch
\newcommand{\hsmaabstractde}
{
In dieser Arbeit wird untersucht, welches Potenzial Position Based Dynamics (PBD)-basierte Echtzeitsimulationen \cite{PBD} in Verbindung mit Game \linebreak Engines für Serious Games im medizinischem Bereich haben. 

Hierfür wird die Unreal Engine 4 zusammen mit NVIDIA FleX \cite{UPP} genutzt, um Software zu untersuchen, die auf einen aktuellen technischen Stand ist. Zuerst wird die Performance und das Simulationsverhalten von NVIDIA FleX  genauer untersucht, um in Erfahrung zu bringen, wie PBD-Simulationen von unterschiedlichen Faktoren beeinflusst werden. Außerdem werden fehlende Funktionalitäten identifiziert, die für die Entwicklung von Serious Games wichtig sind. Daraufhin werden Lösungen präsentiert, mit denen die fehlenden Funktionalitäten implementiert werden können. 
%Dabei wird vor allem auf die umfangreichen Funktionen der Unreal Engine 4 gesetzt. 
So erhalten wir ein umfangreiches Softwareframework, mit dem sich schnell und einfach medizinische Echtzeitsimulationen erstellen lassen.

Außerdem wird eine Lösung präsentiert, mit der sich einfache chirurgische Simulatoren erstellen lassen, ohne Verwendung von haptischer Hardware. Hierfür wird nur eine handelsübliche VR-Anlage (z. B. Oculus Quest 2) mit Motion Controllern benötigt. Dabei kann der Nutzer mithilfe seiner Motion Controller nach virtuellen Werkzeugen greifen und diese intuitiv benutzen. Mithilfe von simulierten Kräften übt ein virtuelles Werkzeug Interaktionskräfte auf die virtuelle Umgebung aus.

Es wird auch eine optisch realistische Nadel-Gewebe-Interaktion umgesetzt. Gewebe und Nadel werden dabei mithilfe von Partikeln und \textit{Shape Matching Constraints} \cite{Shape} modelliert. Die Partikelstruktur des Gewebemodells wird dabei nicht verändert, das heißt, das virtuelle Gewebe wird nicht von der Nadel verletzt beziehungsweise geschnitten.

%Basierend auf Reynolds Boid-System \cite{Reynolds87flocks} wird ein Modell für die Simulation von Tierschwärmen entwickelt. Wesentlicher Fokus liegt dabei auf lebendig wirkende Schwarmbewegungen, interaktives Verhalten und Performance. 
%
%Statt den simulierten Schwarm wie ein Partikelsystem zu behandeln, wird Reynolds Boid-Modell in ein Zellensystem, das mit den zellulären Automaten verwandt ist, transferiert. Diese Anpassung wird vorgenommen, damit die Simulation von herkömmlichen Shadern ausgeführt werden kann.
%
%Das Simulationsmodell unterteilt den Schwarm nicht in einzelne separate Individuen. Stattdessen ist der Schwarm eine im Raum ungleichmäßig verteilte Menge.
%Der Schwarm wird daher als leuchtende Punkte-Wolke, von einer Textur dargestellt.
%
%Die konzipierte Schwarmsimulation wird mittels Content-Driven Multipass Rendering prototypisch implementiert und in Echtzeit ausgeführt.




%Im erstem Teil der Arbeit wird eine Auftriebssimulation für Videospiele entwickelt, um 3D-Modelle auf Wasserflächen schwimmen lassen zu können. Das zugrundeliegende mathematische Modell diese Simulation basiert auf dem archimedischem Prinzip. Die Auftriebssimulation wird mit der Unreal Engine 4 implementiert und ihre integrierte Physik Engine, nvidia PhysX3.3, wird genutzt um Auftriebskräfte auf virtuelle Schwimmkörper wirken zu lassen, welche einem 3D-Modell zugeordnet werden.
%
%Im zweitem Teil der Arbeit wird eine interaktive Wasseroberfläche für Videospiele entwickelt. Lösungsansatz ist hierbei das sogenannte \glqq IWave-Verfahren\grqq{} \cite{tessendorf}. Dieses Verfahren wird mit der Unreal Engine 4 und einer Technik namens \glqq Content Driven Multipass Rendering\grqq{} umgesetzt.

%\cite{tessendorf}
}

% Kurze (maximal halbseitige) Beschreibung, worum es in der Arbeit geht auf Englisch

\newcommand{\hsmaabstracten}
{
This thesis examines the potential of Position Based Dynamics (PBD)-based real-time simulations \cite{PBD} in connection with game engines for serious games with medical background.

For this purpose, the Unreal Engine 4 is used together with NVIDIA FleX \newline \cite{UPP} to examine software that is state-of-the-art. First, the performance and simulation behavior of FleX is examined in more detail in order to find out how FleX simulations are influenced by different factors. Then missing \linebreak functionalities are identified, which could be useful for serious game development. After that, we present solutions which can be used to implement these missing functionalities.
In that way we optain a comprehensive and easy-to-use framework which serves as a development platform for medical real-time simulations.

In addition, a solution is presented how to implement immersive surgical simulators without haptic feedback. Such surgical simulations can be used with standard VR systems, like an Oculus Quest 2, HTC vive or valve index. The user can use his motion controllers to grab virtual tools and use them. With the help of simulated forces, a virtual tool can apply accurate interaction forces to virtual environment.

An optically realistic needle-tissue interaction is also implemented. Tissue and \linebreak needle are modeled with the help of particles and \textit{Shape Matching Constraints} \newline \cite{Shape}. The particle structure of the tissue model is not changed.
}



