% Die Arbeit besteht aus Kapiteln (chapter)
% strg + t -> commend selection. strg+ u -> uncommend selection
\chapter{Einleitung}
\label{chap_Einleitung}



\section{Motivation und Problemstellung}


% Motivation für Serous Games:
Die klassische Art, Chirurgie praktisch zu unterrichten, ist die Beobachtung eines erfahrenen Chirurgen über mehrere durchgeführte Operationen. Dieser Ansatz wird zunehmend durch rechtliche und ethische Bedenken hinsichtlich der Patientensicherheit, des Zeitbedarfs erfahrener Chirurgen und der Kosten für die OP-Zeit behindert \cite{SurgSim}.

Deswegen sind Serius Games in der Medizin wichtige Hilfsmittel um Personal auszubilden und zu trainieren. Vor allem in der Chirurgie werden immer häufiger medizinische Echtzeitsimulationen eingesetzt um Personal auf Operationen vorzubereiten \cite{SimRole}. So kann das Personal besser auf eine Operation am echten Patienten vorbereitet werden und die Chance, dass die Operation fehlerfrei verläuft, kann gesteigert werden \cite{VRNeuro}.

% Motivation VR Serous Games:
Mithilfe von handelsüblichen VR-Geräten und haptischem Feedback können Echtzeitsimulatoren erstellt werden, die besonders immersiv und realistisch wirken. Validierungsstudienergebnisse zeigen, dass solche VR-Anwendungen gut geeignet sind, um medizinische Fachkräfte effektiv für gezielte Operationsverfahren auszubilden \cite{VRSim20}.

%Aufgrund der erfolgreichen Entwicklung von chirurgischen Echtzeitsimulatoren wird in dieser Arbeit un

%Motivation PBD:

LESEZEICHEN!!! Weiter mit Motivation für \ac{PBD}!!!!!

Beliebt im medizinischem bereich weil:

- Einfach (einfacher zu implementieren und zu erweitern auch weil wirs mit punktmassen zutun haben, siehe kapitel xxx!), hohe Effizienz, Stabilität, Echtzeitleistung, 

im Gegensatz zu Kraftbasierte methoden: PBD projiziert Position direkt als Lösung (lösung von constraints) statt Beschleunigungen zu integrieren (um equilibrium confi zu erreichen). Dadurch beseitigt PBD das Überschießproblem. PBD ist nicht so genau wir kraftbasierte Methoden

Kann visuell plausible Ergebnisse liefern \cite{PBDKidney} \cite{BreastBiopsy}, obwohl ungenau .

Wurde schon erfolgreich eingesetzt für:

Simulation von chirurgischen Fäden (nähen, mit haptic feedback) \cite{PBDThread}, 
der Simulation von laparoskopischen Operationen in VR \cite{VRLaparoscop}, 
dem chirurgischen Schneiden \cite{PBDCutting}
und der Simulation von Roboteroperationen \cite{VRRobSim} (auch in VR).

\section{Einordnung und Abgrenzung}

Sehr ähnlich zu \cite{VRSim20}. Unterschied: In \cite{VRSim20} wirkt sich Reaktionskraft auf echtes Werkzeug (haptische Hardware) aus. Hier wird Reaktionskraft auf virtuelles Werkzeug angewandt und Spieler erhält nur visuelles Force Feedback. 

Ist dann zwar Weniger realistisch (das ist natürlich unbestreitbar), dafür kann Software mit jedem handelsüblichen VR-Gerät genutzt werden. Zusätzliche haptische Hardware wird nicht benötigt. Gute VR-Geräte gibts schon für 300 euro, Jede Privatperson die ein VR-Gerät mit PC oder gar PS-VR hat, könnte Trainings-Software nutzen.

Habe Für Gewebe SelfCollision verboten, im Gegensatz zu vielen anderen Arbeiten (\cite{BreastBiopsy} \cite{PBDKidney}).

Einfache Nadel-Gewebe-Interaktion ohne Verletzung oder schneiden, mit virtuellen Kräften auf Nadel (Reibungs und Klemmkräfte, Interaktionskraft usw)


%Falls bei präsi gefragt: Wie wird kann force feedback berrechnet werden? Siehe Fig. 7 aus \cite{VRSim20} virtuelles Lyaryngoskop (proxy) drückt Partikel ein (Partikel sehen Proxy-Werkzeug als statische Wand die sich nicht verschieben lässt). Vektoren zu Rest-positionen werden gespannt, Durchschnitt, multiplizieren mit stiffnes-faktor für das Gewebe. Genau so hab ichs am Anfang ja auch gemacht, das war aber zu aufwändig und inperformant.

\section{Aufbau dieser Arbeit}
%... mach ich am Ende...

...Mach ich zum Schluss, aber nur wenn es die Betreuer auch haben möchten...





