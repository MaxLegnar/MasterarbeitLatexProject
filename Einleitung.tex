% Die Arbeit besteht aus Kapiteln (chapter)
% strg + t -> commend selection. strg+ u -> uncommend selection
\chapter{Einleitung}
\label{chap_Einleitung}



\section{Motivation und Problemstellung}


% Motivation für Serous Games:
Die klassische Art, bestimmte Operationen praktisch zu unterrichten, ist die Beobachtung eines erfahrenen Chirurgen über mehrere durchgeführte Operationen. Dieser Ansatz wird zunehmend durch rechtliche und ethische Bedenken hinsichtlich der Patientensicherheit, des Zeitbedarfs und der Kosten für die OP-Zeit behindert \cite{SurgSim}.

Deswegen sind Serious Games in der Medizin wichtige Hilfsmittel um Personal auszubilden und zu trainieren. Vor allem in der Chirurgie werden immer häufiger medizinische Echtzeitsimulationen eingesetzt um Personal auf Operationen vorzubereiten \cite{SimRole}. So kann das Personal besser auf eine Operation am echten Patienten vorbereitet werden und die Chance, dass die Operation fehlerfrei verläuft, kann gesteigert werden \cite{VRNeuro}.

%Motivation PBD:
Die Simulationstechnik \ac{PBD} \cite{PBD} hat sich bereits in vielen Arbeiten als vielversprechende Simulationstechnik für chirurgische Simulatoren erwiesen (siehe Kapitel \ref{section_PBD} für einige Beispiele). 

%Motivation PBD und Überleitung zu Problemstellung 1:
Aufgrund der guten Stabilität, Einfachheit und Effizienz von PBD wird auch in dieser Arbeit eine PBD-basierte Physik Engine (NVIDIA FleX \cite{UPP}) genutzt um Gewebe, Flüssigkeiten und Rigid Bodies in Echtzeit zu simulieren. Zusätzlich wird eine Game Engine (Unreal Engine 4 \cite{UE4FlexDoc}) eingesetzt, die fast alle restlichen Elemente implementiert, die für Serious Games im chirurgischem Bereich benötigt werden. Dabei werden einige zusätzliche Elemente methodisch entwickelt und implementiert um die vorhandene Game- und Physik-Engine um zusätzliche Funktionalitäten zu erweitern, die für chirurgische Simulatoren nützlich sein könnten (Kapitel \ref{chap_Flex_Engine}).

% Motivation VR Serous Games:
\ac{VR} spielt für Serous Games auch eine wichtige Rolle. Mit einem handelsüblichen \ac{HMD} kann die virtuelle Umgebung nämlich besonders immersiv dargestellt werden.
Mithilfe von VR-Geräten und haptischem Feedback können Echtzeitsimulatoren erstellt werden, die besonders immersiv und realistisch wirken. 
Studien zeigen, dass VR-Anwendungen mit haptischem Feedback gut geeignet sind, um medizinische Fachkräfte effektiv für bestimmte Operationsverfahren auszubilden \cite{VRSim20} \cite{VRHapticSim}.


\cite{VRHapticSim}

% Überleitung problemstellung 2:
In dieser Arbeit wird hingegen eine Methode präsentiert, mit der chirurgische VR-Simulatoren ohne haptisches Feedback erstellt werden können. Der Nutzer benötigt nur eine handelsübliche VR-Anlage 
%mit Handtracking oder Motion Controllern 
um unterschiedliche Operationen trainieren zu können. 
%Für diese Arbeit wurde eine Oculus Quest 2 für 300 Euro eingesetzt. 
Dies ermöglicht es Nutzern, die keinen Zugriff auf haptische Hardware haben, Operationen in virtueller Realität zu trainieren. 

Eine besondere Herausforderung ist dabei eine Methode zu finden, mit der der Nutzer möglichst realistisch und präzise mit der virtuellen Umgebung interagieren kann, ohne haptisches Feedback zu erhalten. In dieser Arbeit wurde hierfür ein System entwickelt, das es dem Spieler ermöglicht mit virtuellen Werkzeugen simulierte Kräfte auf die virtuelle Umgebung auszuüben. So kann der Spieler beispielsweise eine Nadel in simuliertes Gewebe einführen, indem er seine echte Hand auf bestimmte Weise positioniert und beschleunigt und dadurch Interaktionskräfte zwischen Nadel und Gewebe erzeugt. 
Dabei erhält der Nutzer nur visuelles Kräftefeedback und kein haptisches.
Die Nadel-Gewebe Interaktion mit rein virtuellen Kräften ist dabei auch eine besondere Herausforderung, die in Kapitel \ref{sec_needle} behandelt wird. 
Bei der Nadel-Gewebe-Interaktion wird das Partikelmodel des Gewebes nicht verändert. 



%Um am Ende dieser Arbeit ein nützliches Framework zu erhalten, das für die Entwicklung von chirurgischen Simulatoren geeignet ist, wird analysiert welche weiteren Funktionalitäten für die Entwicklung von chirurgischen Simulatoren nützlich sind. Diese Funktionalitäten werden dann methodisch 
%um photorealistische Bilder zu zeichnen und von ihren vielseitigen Funktionalitäten zu profitieren, welche gut für die Entwicklung von Serious Games im chirurgischem Bereich geeignet sind.

%Außerdem wird in dieser Arbeit eine Game Engine, die Unreal Engine 4, genutzt um von ihrer effizienten Kollisionserkennung, ihren Renderfähigkeiten und vielseitigen Software-Bibliotheken zu profitieren. 

%In dieser Arbeit werden Konzepte vorgestellt, mit denen PBD basierte Physik Engines zusammen mit Game Engines 
%In dieser Arbeit wird mithilfe einer Game Engine und einer PBD basierten Physik Engine 



%\section{Einordnung und Abgrenzung}

%Sehr ähnlich zu \cite{VRSim20}. Unterschied: In \cite{VRSim20} wirkt sich Reaktionskraft auf echtes Werkzeug (haptische Hardware) aus. Hier wird Reaktionskraft auf virtuelles Werkzeug angewandt und Spieler erhält nur visuelles Force Feedback. 

%Ist dann zwar Weniger realistisch (das ist natürlich unbestreitbar), dafür kann Software mit jedem handelsüblichen VR-Gerät genutzt werden. Zusätzliche haptische Hardware wird nicht benötigt. Gute VR-Geräte gibts schon für 300 euro, Jede Privatperson die ein VR-Gerät mit PC oder gar PS-VR hat, könnte Trainings-Software nutzen.

%Habe Für Gewebe SelfCollision verboten, im Gegensatz zu vielen anderen Arbeiten (\cite{BreastBiopsy} \cite{PBDKidney}).

%Einfache Nadel-Gewebe-Interaktion ohne Verletzung oder schneiden, mit virtuellen Kräften auf Nadel (Reibungs und Klemmkräfte, Interaktionskraft usw)


%Falls bei präsi gefragt: Wie wird kann force feedback berrechnet werden? Siehe Fig. 7 aus \cite{VRSim20} virtuelles Lyaryngoskop (proxy) drückt Partikel ein (Partikel sehen Proxy-Werkzeug als statische Wand die sich nicht verschieben lässt). Vektoren zu Rest-positionen werden gespannt, Durchschnitt, multiplizieren mit stiffnes-faktor für das Gewebe. Genau so hab ichs am Anfang ja auch gemacht, das war aber zu aufwändig und inperformant.

\section{Aufbau dieser Arbeit}
%... mach ich am Ende...

...Mach ich zum Schluss, aber nur wenn es die Betreuer auch haben möchten...





