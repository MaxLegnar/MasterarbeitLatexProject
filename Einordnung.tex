\chapter{Einordnung dieser Arbeit}
\label{chap_Einordnung}

Was genau mache ich anders im Vergleich zu anderen Werken? Auf welche Arbeiten baue ich auf? Diesen Teil mach ich eher gegen ende, weil ich noch nicht ganz genau weiss wo das hinführen wird... 

Vergleichbare Arbeiten wären sehr wahrscheinlich: \cite{Oliver14Tschesche} - Aber wegen meiner neuen Herangehensweise, den Schwarm als Dichte/Masse zu betrachten, gibt es nicht viele Vergleiche bisher.

Hier wird auch derzeitiger technischer Stand (Reynolds und CDMR) mit den wichtigsten Papern aufgeführt. Knifflig: Es gibt kaum/kein Paper über CDMR. 


%
%Im Rahmen dieser Arbeit wurde eine Simulation einer interaktiven Wasseroberfläche für Videospiele implementiert. Lösungsansatz ist hierbei das sogenannte \glqq IWave-Verfahren\grqq{} von Tessendorf \cite{tessendorf}. Wesentlicher Bestandteil dieses Verfahrens ist die Lösung einer Differenzialgleichung mithilfe einer Faltungsoperation. In \cite{Gems4} wird empfohlen, diese Art der Faltung in eine für SIMD-Pipeline geeignete Form zu bringen, damit die Faltung effizient durchgeführt wird. In dieser Arbeit wurde stattdessen das gesamte IWave-Verfahren mit einer Technik namens \ac{CDMR} umgesetzt. Dies ist eine einfachere Möglichkeit das IWave-Verfahren mithilfe der Unreal Engine 4 zu implementieren.
%
%Während der Entwicklung dieses interaktiven Wasser-Shaders wurde eine Musterlösung einer solchen Wasseroberfläche von Epic Games, den Machern der Unreal Engine, veröffentlicht \cite{FluidSurface}. Der mathematische Ansatz ist bei deren Lösung ein Anderer als der dieser Arbeit.
%
%Statt, wie beim IWave-Verfahren, die Wasserbewegung, basierend auf vereinfachten Navier-Stokes-Gleichungen, zu berechnen, wird bei der Lösung von Epic Games ein simplerer Ansatz verfolgt. Es wird im Wesentlichen versucht eine Animation zu erzeugen, welche an ringförmige, sich ausbreitende Wellen erinnert. Naturgetreue, physikalische Zusammenhänge werden bei dieser Lösungsmethode nicht berücksichtigt. Statt eine Differenzialgleichung zu lösen, wird die gewünschte Wasserwelle mit einer einfachen Bewegungsfunktion beschrieben. In Kapitel \ref{sec_VergleichIWave} werden die beiden Lösungsmethoden subjektiv miteinander verglichen.


